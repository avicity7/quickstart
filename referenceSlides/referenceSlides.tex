\documentclass{beamer}
\hypersetup{pdfpagemode=FullScreen}
\usepackage{fontspec}
\usepackage{minted}

\setsansfont{Open Sans}

\usetheme{metropolis}           % Use metropolis theme
\title{List Comprehension}
\subtitle{in Python}
\date{\today}
\author{Karl Steven Velasco Orjalo}
\institute{Python Quickstart}

\begin{document}
\metroset{block=fill}
  \maketitle

  \begin{frame}{Lesson Outline}
    \setbeamertemplate{section in toc}[sections numbered]
    \tableofcontents
  \end{frame}

  \section{Basic Usage}
  \begin{frame}[fragile]
    \frametitle{Basic Usage}

    Here's how we can use list comprehension to go through a list in Python.
    \newline
    \begin{minted}{python}
      lst = [1,2,3,4]

      for num in lst: 
        print(num)
    \end{minted}
  \end{frame}

  \section{Using it with other things}
  \begin{frame}[fragile]
    \frametitle{With Dictionaries}

    We can use it with \textbf{dictionaries} too.
    \newline

    \begin{minted}{python3}
      dct = {"age":3}
      arr = [dct]

      print(arr)
    \end{minted}
    \vspace{0.8cm}
    \begin{block}{Be careful!}
      The output of this would be an array with length of one, not two!
    \end{block}

  \end{frame}
  
  \appendix

  \begin{frame}[standout]
    Questions?
  \end{frame}

\end{document}