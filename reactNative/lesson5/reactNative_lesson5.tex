\documentclass{beamer}
\hypersetup{pdfpagemode=FullScreen}
\usepackage{fontspec}
\usepackage{tcolorbox}
\tcbuselibrary{minted,skins}
\usepackage{graphicx}
\graphicspath{ {./images/} }

\newtcblisting{jscode}{
  listing engine=minted,
  colback=codebg,
  colframe=codebg,
  listing only,
  minted style=monokai,
  minted language=jsx,
  left=1mm,
}
\newtcblisting{jscodesmall}{
  listing engine=minted,
  colback=codebg,
  colframe=codebg,
  listing only,
  minted style=monokai,
  minted language=jsx,
  minted options={fontsize=\tiny},
  left=1mm,
}
\newtcblisting{bashcode}{
  listing engine=minted,
  colback=bashbg,
  colframe=bashbg,
  listing only,
  minted style=native,
  minted language=bash,
  minted options={linenos=true,texcl=true},
  left=1mm,
}
\definecolor{codebg}{rgb}{0.2,0.2,0.2}
\definecolor{bashbg}{rgb}{0.85,0.85,0.85}

\setsansfont{Open Sans}

\usetheme{metropolis}           % Use metropolis theme
\title{Lesson 5: Diving Deeper into React Native}
\subtitle{Creating a Feed}
\date{\today}
\author{Karl Steven Velasco Orjalo}
\institute{React Native Quickstart}

\begin{document}
\metroset{block=fill}
  \maketitle

  \begin{frame}{Lesson Outline}
    \setbeamertemplate{section in toc}[sections numbered]
    \tableofcontents
  \end{frame}

  \section{Exploring JSON, JavaScript Object Notation}
  \begin{frame}[fragile]
    \frametitle{Getting the news}
    We'll be using an online service to convert the New York Times' RSS feed to a JSON object. 

    It is accessible through the link \url{https://newsapi.org/v2/top-headlines?country=us&apiKey=7c185ff7708e44dea7b0f22f482203d5}.
  \end{frame}
  \begin{frame}[fragile]
    \frametitle{Getting the news (cont.)}
    Under \verb|items|, we can see that we now have an array of articles. The format of each article is as below: 

    \begin{jscodesmall}
{
  "source": {
      "id": null,
      "name": "sourceName"
  },
  "author": "Author Name",
  "title": "Article Title",
  "description": "Article Description",
  "url": "uri",
  "urlToImage": "uri",
  "publishedAt": "Publish Date",
  "content": null
},
    \end{jscodesmall}
  \end{frame}
  
  \section{Creating an Article thumbnail}
  \begin{frame}[fragile]
    \frametitle{Displaying a single Article}
    In \verb|./src/components/|, create a file called \verb|articleThumbnail.jsx|.

    Enter the following code: 

    \begin{jscodesmall}
import React from 'react'
import {View, Text} from 'react-native'

const ArticleThumbnail = ({item}) => {
  return(
    <View>
      <Text>{item.title != null ? item.title : ''}</Text>
      <Text>{item.publishedAt != null ? item.publishedAt : ''}</Text>
      <Text>{item.author != null ? item.author : ''}</Text>
    </View>
  )
}

export default ArticleThumbnail
    \end{jscodesmall}
  \end{frame}
  \begin{frame}[fragile]
    \frametitle{Displaying a single Article (cont.)}
    We'll now bring in our Component into \verb|./src/pages/Home.jsx|. 

    \begin{jscodesmall}
import ArticleThumbnail from '../components/articleThumbnail'

const Home = () => {
  return (
    <View style={globalStyles.container}>
      <Button onPress={()=>{console.log("Button pressed!")}} />
      <ArticleThumbnail />
    </View>
  )
}
    \end{jscodesmall}
  \end{frame}
  \begin{frame}[fragile]
    \frametitle{Displaying a single Article (cont.)}
    Next, create an object called \verb|newsItem| with a single article. 
    You can choose any from the provided link a couple slides back.

    Put the following code above your \verb|Home| const.

    \begin{jscodesmall}
const newsItem = {
  "source": {
      "id": null,
      "name": "sourceName"
  },
  "author": "Author Name",
  "title": "Article Title",
  "description": "Article Description",
  "url": "uri",
  "urlToImage": "uri",
  "publishedAt": "Publish Date",
  "content": null
}
    \end{jscodesmall}
  \end{frame}
  \begin{frame}[fragile]
    \frametitle{Displaying a single Article (cont.)}
    Now, let's give \verb|newsItem| to our \verb|<ArticleThumbnail />|.

    \begin{jscode}
<ArticleThumbnail item={newsItem} />
    \end{jscode}
  \end{frame}

  \section{Displaying multiple Articles}
  \begin{frame}[fragile]
    \frametitle{Having data with multiple objects}
    In \verb|./src/pages/Home.jsx|, add 2 more articles to your \verb|newsItem|.
    
    Hint: \verb|newsItem| is now an Array of Objects!
  \end{frame}
  \begin{frame}[fragile]
    \frametitle{Rendering multiple ArticleThumbnails}
    We are able to make as many \verb|<ArticleThumbnail />| instances as needed through a \verb|<Flatlist />|. 

    Replace your singlular \verb|<ArticleThumbnail />| with a \verb|<FlatList />|. 

    Make sure to import \verb|FlatList| from \verb|'react-native'|!

    \begin{jscodesmall}
import {View, FlatList} from 'react-native'

<FlatList
  data={newsItem}
  renderItem={({ item }) => <ArticleThumbnail item={item} />}
/>
    \end{jscodesmall}
  \end{frame}

  \begin{frame}[fragile]
    \frametitle{Explaining <FlatList/>}
    \verb|<FlatList />| has two required components, the \verb|data| and \verb|renderItem|. 

    \verb|data| must be a list/Array, and the objects of the list would be used for the \verb|renderItem|.

    In the \verb|renderItem|, we must create a function that takes in the individual object of the list.
    The provided code has this named \verb|item|, but can be renamed to better fit the project as required. 
  \end{frame}
  \begin{frame}[fragile]
    \frametitle{Explaining <FlatList/> (cont.)}
    \verb|<FlatList />| has several other parameters we can provide. 

    Useful ones include \verb|ListHeaderComponent| and \verb|extradata|. 

    Refer to documentation for more details: \url{https://reactnative.dev/docs/flatlist}.

  \end{frame}

  \begin{frame}[fragile]
    \frametitle{Fixing our styling}
    The \verb|<Button />| is blocking our \verb|FlatList| when we attempt to scroll down. 

    Where should we move our \verb|<Button />| instead? 
  \end{frame}
  \begin{frame}[fragile]
    \frametitle{Fixing our styling (cont.)}
    Now as you can see, we have some white space when scrolling our \verb|FlatList| down. 

    Find out which of the style properties in \verb|./src/styles/globalStyles| is causing this behaviour. 
  \end{frame}

  \section{Making everything look good}
  \begin{frame}[fragile]
    \frametitle{Styling our ArticleThumbnail}
    On your own, find a mobile app that displays Articles and copy their style. (20mins)

    Feel free to ask any questions!
  \end{frame}

  \appendix

  \begin{frame}[standout]
    End of Lesson

    {\small Questions? Reach out at:}
    {\footnotesize karlorjalo@gmail.com}
  \end{frame}

\end{document}