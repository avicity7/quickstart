\documentclass{beamer}
\hypersetup{pdfpagemode=FullScreen}
\usepackage{fontspec}
\usepackage{tcolorbox}
\tcbuselibrary{minted,skins}
\usepackage{graphicx}
\graphicspath{ {./images/} }

\newtcblisting{jscode}{
  listing engine=minted,
  colback=codebg,
  colframe=codebg,
  listing only,
  minted style=monokai,
  minted language=jsx,
  left=1mm,
}
\newtcblisting{jscodesmall}{
  listing engine=minted,
  colback=codebg,
  colframe=codebg,
  listing only,
  minted style=monokai,
  minted language=jsx,
  minted options={fontsize=\tiny},
  left=1mm,
}
\newtcblisting{bashcode}{
  listing engine=minted,
  colback=bashbg,
  colframe=bashbg,
  listing only,
  minted style=native,
  minted language=bash,
  minted options={linenos=true,texcl=true},
  left=1mm,
}
\definecolor{codebg}{rgb}{0.2,0.2,0.2}
\definecolor{bashbg}{rgb}{0.85,0.85,0.85}

\setsansfont{Open Sans}

\usetheme{metropolis}           % Use metropolis theme
\title{Lesson 4: Exploring React Native}
\subtitle{Creating a Component}
\date{\today}
\author{Karl Steven Velasco Orjalo}
\institute{React Native Quickstart}

\begin{document}
\metroset{block=fill}
  \maketitle

  \begin{frame}{Lesson Outline}
    \setbeamertemplate{section in toc}[sections numbered]
    \tableofcontents
  \end{frame}

  \section{Creating a basic Button}
  \begin{frame}[fragile]
    \frametitle{Adding a button to a page}
    We'll now be working on \verb|./src/pages/Home.jsx| to make a button and \verb|./src/styles/global.js| to style the button. 
  \end{frame}
  \begin{frame}[fragile]
    \frametitle{Adding a button to a page (cont.)}
    In \verb|./src/pages/Home.jsx| we will now make a basic button. 
    First, let's bring in \verb|Pressable| from \verb|'react-native'| as such: 

    \begin{jscodesmall}
import {View, Text, Pressable} from 'react-native'
    \end{jscodesmall}
  \end{frame}
  \begin{frame}[fragile]
    \frametitle{Adding a button to a page (cont.)}
    Then, let's replace our existing \verb|<Text>| with a \verb|<Pressable>| and a \verb|<Text>| child as such: 

    \begin{jscodesmall}
return (
  <View style={globalStyles.container}>

    <Pressable>
        <View>
          <Text>Press me!</Text>
        </View>
    </Pressable>

  </View>
)
    \end{jscodesmall}
  \end{frame}

  \begin{frame}[fragile]
    \frametitle{Styling our Button}
    Let's now style our button in \verb|./src/styles/global.js|!

    \vspace{0.5cm}
    \begin{jscodesmall}
publishButton: {
  backgroundColor: "black",
  borderRadius: 15,
  marginLeft: 20,
  marginRight: 20,
  marginBottom: 50
},
    \end{jscodesmall}
  \end{frame}
  \begin{frame}[fragile]
    \frametitle{Styling our Button (cont.)}
    Let's now style our button in \verb|./src/styles/global.js|!

    \vspace{0.5cm}
    \begin{jscodesmall}
button: {
  backgroundColor: "pink",
  padding: 8,
  borderRadius: 8,
},
    \end{jscodesmall}
  \end{frame}
  \begin{frame}[fragile]
    \frametitle{Styling our Button (cont.)}
    Now, let's use the style in \verb|.src/pages/Home.jsx|.

    \vspace{0.5cm}
    \begin{jscodesmall}
return (
  <View style={globalStyles.container}>

    <Pressable>
        <View style={globalStyles.button}>
          <Text>Press me!</Text>
        </View>
    </Pressable>

  </View>
)
    \end{jscodesmall}
  \end{frame}

  \begin{frame}[fragile]
    \frametitle{Giving our button functionality}
    To give our button functionality, we have to edit our \verb|<Pressable>|. 

    \vspace{0.5cm}
    \begin{jscodesmall}
return (
  <View style={globalStyles.container}>
    <Pressable onPress={()=>{console.log("Button pressed!")}}>
      <View style={globalStyles.button}>
        <Text>Press me!</Text>
      </View>
    </Pressable>
  </View>
)
    \end{jscodesmall}
  \end{frame}

  \section{Making our Button a reusable Component}
  \begin{frame}[fragile]
    \frametitle{Why do we need reusable components?}
    
    If you need multiple Buttons in multiple pages, we're not going to copy/paste the same code several times.
    
    Instead, we'll make it into an easily reusable Component. 
  \end{frame}

  \begin{frame}[fragile]
    \frametitle{Making a component file}
    In \verb|./src/components/|, create a file called \verb|button.jsx|. 

    \vspace{0.5cm}
    \begin{jscodesmall}
import React from 'react'
import {View, Text, Pressable} from 'react-native'
import globalStyles from "../styles/globalStyles";

const Button = ({onPress}) => {
  return (
    <Pressable onPress={onPress}>
      <View style={globalStyles.button}>
        <Text>Press me!</Text>
      </View>
    </Pressable>
  )
}

export default Button
    \end{jscodesmall}
  \end{frame}

  \begin{frame}[fragile]
    \frametitle{Using our component}
    Back in \verb|./src/pages/Home.jsx|, import our newly created \verb|Button|.

    \vspace{0.5cm}
    \begin{jscodesmall}
import Button from '../components/Button'

const Home = () => {
  return (
    <View style={globalStyles.container}>
      <Button onPress={()=>{console.log("Button pressed!")}} />
    </View>
  )
}
    \end{jscodesmall}
  \end{frame}

  \appendix

  \begin{frame}[standout]
    End of Lesson

    {\small Questions? Reach out at:}
    {\footnotesize karlorjalo@gmail.com}
  \end{frame}

\end{document}