\documentclass{beamer}
\hypersetup{pdfpagemode=FullScreen}
\usepackage{fontspec}
\usepackage{tcolorbox}
\tcbuselibrary{minted,skins}
\usepackage{graphicx}
\graphicspath{ {./images/} }

\newtcblisting{jscode}{
  listing engine=minted,
  colback=codebg,
  colframe=codebg,
  listing only,
  minted style=monokai,
  minted language=jsx,
  left=1mm,
}
\newtcblisting{jscodesmall}{
  listing engine=minted,
  colback=codebg,
  colframe=codebg,
  listing only,
  minted style=monokai,
  minted language=jsx,
  minted options={fontsize=\tiny},
  left=1mm,
}
\newtcblisting{bashcode}{
  listing engine=minted,
  colback=bashbg,
  colframe=bashbg,
  listing only,
  minted style=native,
  minted language=bash,
  minted options={linenos=true,texcl=true},
  left=1mm,
}
\definecolor{codebg}{rgb}{0.2,0.2,0.2}
\definecolor{bashbg}{rgb}{0.85,0.85,0.85}

\setsansfont{Open Sans}

\usetheme{metropolis}           % Use metropolis theme
\title{Lesson 8: Becoming a React Native native}
\subtitle{Auth, Text Input, Pages and more}
\date{\today}
\author{Karl Steven Velasco Orjalo}
\institute{React Native Quickstart}

\begin{document}
\metroset{block=fill}
  \maketitle

  \begin{frame}{Lesson Outline}
    \setbeamertemplate{section in toc}[sections numbered]
    \tableofcontents
  \end{frame}

  \section{Firebase Auth}
  \begin{frame}[fragile]
    \frametitle{Handling user sign-ins}
    To handle user authentication as well as Sign-up/in, we'll just use Firebase's Authentication service. 

    Go to to the Firebase console and find `Authentication' under the Build tab on the left. 
    Create an authentication service with \verb|Email/Password|.
  \end{frame}
  \begin{frame}[fragile]
    \frametitle{Handling user sign-ins (cont.)}
    In our app, we have to have at least 2 inputs, an email and a password as well as a Sign-up button. 

    In \verb|./src/pages/Account.jsx| import \verb|TextInput| and \verb|Pressable|. 

    \begin{jscodesmall}
import { View, FlatList, Text, TextInput, Pressable } from 'react-native'
    \end{jscodesmall}
  \end{frame}
  \begin{frame}[fragile]
    \frametitle{Handling user sign-ins (cont.)}
    Next we can use the \verb|useState()| hook to store our email and password. 

    \begin{jscodesmall}
const [email,setEmail] = useState('')
const [password,setPassword] = useState('')
    \end{jscodesmall}
  \end{frame}
  \begin{frame}[fragile]
    \frametitle{Handling user sign-ins (cont.)}
    We'll need some functions from \verb|firebase/auth| so import them. 

    \begin{jscodesmall}
import { getAuth, createUserWithEmailAndPassword } from "firebase/auth";
    \end{jscodesmall}

    As you can probably guess, we now have a function to check the current auth-ed user, create a new account and sign-in. 
  \end{frame}
  \begin{frame}[fragile]
    \frametitle{Handling user sign-ins (cont.)}
    Now make two pairs of \verb|<Text />| and \verb|<TextInput />|, one for email and one for password. 

    \begin{jscodesmall}
<Text>Email</Text>
<TextInput
        placeholder='email'
        textContentType='emailAddress'
        onChangeText={newText => setEmail(newText)}
    />
<Text>Password</Text>
<TextInput
    placeholder = 'password'
    textContentType='password'
    onChangeText={newText => setPassword(newText)}
/>
    \end{jscodesmall}

    As you can probably guess, we now have a function to check the current auth-ed user, create a new account and sign-in.
  \end{frame}
  \begin{frame}[fragile]
    \frametitle{Handling user sign-ins (cont.)}
    Then call the \verb|createUserWithEmailAndPassword()| function with a \verb|<Pressable />|. 

    \begin{jscodesmall}
<Pressable 
onPress = {()=>{
  createUserWithEmailAndPassword(auth, email, password)
    .then((userCredential) => { 
        const user = userCredential.user
        console.log("User created with email: "+user.email)
    })
    .catch((error) => {
        const errorCode = error.code;
        const errorMessage = error.message;
    });
}}
>
    <Text style = {{fontSize: 16, color: "black"}}> Sign Up </Text>
</Pressable>
    \end{jscodesmall}
  \end{frame}

    \section{Making nested pages}

    \begin{frame}[fragile]
      \frametitle{Making a Stack of Pages}
      In order to push the user into an info page after they've signed up, we can create a Stack of Pages. 

      First, import the following function as such, and \verb|yarn add| the two modules needed. (Small challenge!)

      \begin{jscodesmall}
import {createNativeStackNavigator} from '@react-navigation/native-stack';
      \end{jscodesmall}
    \end{frame}
    \begin{frame}[fragile]
      \frametitle{Making a Stack of Pages (cont.)}
      Initialize the Stack with the following code by calling \verb|createNativeStackNavigator|:

      \begin{jscode}
const Stack = createNativeStackNavigator();
      \end{jscode}
    \end{frame}
    \begin{frame}[fragile]
      \frametitle{Making a Stack of Pages (cont.)}
      Rename the existing \verb|const Account| into \verb|const AccountScreen|. 

      Above \verb|const AccountScreen|, create the following page: 

      \begin{jscodesmall}
const LoggedInScreen = () => {
  const auth = getAuth()

  return( 
    <Text>{auth.currentUser.email}</Text>
  )
}
      \end{jscodesmall}
    \end{frame}
    \begin{frame}[fragile]
      \frametitle{Making a Stack of Pages (cont.)}
      Below \verb|const AccountScreen| create a \verb|const Account| with the following code: 

      \begin{jscodesmall}
const Account = () => {
  return (
    <Stack.Navigator>
      <Stack.Screen
      name = "AccountScreen"
      component={AccountScreen}
      options={{ headerShown: false }}
      />
      <Stack.Screen
      name = "LoggedInScreen"
      component={LoggedInScreen}
      options={{ headerShown: false }}
      />
    </Stack.Navigator>
  )
}        
      \end{jscodesmall}
    \end{frame}
    \begin{frame}[fragile]
      \frametitle{Making a Stack of Pages (cont.)}
      Below the \verb|console.log()| put the following code:

      \begin{jscodesmall}
navigation.navigate("LoggedInScreen")
      \end{jscodesmall}
    \end{frame}

  \appendix

  \begin{frame}[standout]
    End of Lesson

    {\small Questions? Reach out at:}
    {\footnotesize karlorjalo@gmail.com}
  \end{frame}

\end{document}