\documentclass{beamer}
\hypersetup{pdfpagemode=FullScreen}
\usepackage{fontspec}
\usepackage{tcolorbox}
\tcbuselibrary{minted,skins}
\usepackage{graphicx}
\graphicspath{ {./images/} }

\newtcblisting{jscode}{
  listing engine=minted,
  colback=codebg,
  colframe=codebg,
  listing only,
  minted style=monokai,
  minted language=jsx,
  left=1mm,
}
\newtcblisting{jscodesmall}{
  listing engine=minted,
  colback=codebg,
  colframe=codebg,
  listing only,
  minted style=monokai,
  minted language=jsx,
  minted options={fontsize=\tiny},
  left=1mm,
}
\newtcblisting{bashcode}{
  listing engine=minted,
  colback=bashbg,
  colframe=bashbg,
  listing only,
  minted style=native,
  minted language=bash,
  minted options={linenos=true,texcl=true},
  left=1mm,
}
\definecolor{codebg}{rgb}{0.2,0.2,0.2}
\definecolor{bashbg}{rgb}{0.85,0.85,0.85}

\setsansfont{Open Sans}

\usetheme{metropolis}           % Use metropolis theme
\title{Lesson 6: Diving Deeper into React Native}
\subtitle{Bringing in external data}
\date{\today}
\author{Karl Steven Velasco Orjalo}
\institute{React Native Quickstart}

\begin{document}
\metroset{block=fill}
  \maketitle

  \begin{frame}{Lesson Outline}
    \setbeamertemplate{section in toc}[sections numbered]
    \tableofcontents
  \end{frame}

  \section{Getting data from an online source}
  \begin{frame}[fragile]
    \frametitle{Getting the news}
    We'll be using an online service to convert the New York Times' RSS feed to a JSON object. 

    It is accessible through the link \url{https://newsapi.org/v2/top-headlines?country=us&apiKey=7c185ff7708e44dea7b0f22f482203d5}.
  \end{frame}
  \begin{frame}[fragile]
    \frametitle{The Fetch API}
    In JavaScript, we can use the \verb|fetch()| API to do GET and POST operations.

    An example of how we would use it in React Native as a \verb|getData()| function: 

    \begin{jscodesmall}
const getData = async() => {
  const response = await fetch('https://newsapi.org/v2/top-headlines?country=us&apiKey=7c185ff7708e44dea7b0f22f482203d5')
  const parsedResponse = await response.json()
  const articles = parsedResponse.articles
  console.log(articles)
}  
    \end{jscodesmall}
  \end{frame}
  \begin{frame}[fragile]
    \frametitle{The Fetch API}
    We can now use this function in \verb|./src/pages/Home.jsx|.

    Put it the function \verb|getData()| above the \verb|Home| const. 
    \begin{jscodesmall}
const getData = async() => {
  const response = await fetch('https://newsapi.org/v2/top-headlines?country=us&apiKey=7c185ff7708e44dea7b0f22f482203d5')
  const parsedResponse = await response.json()
  const articles = parsedResponse.articles
  console.log(articles)
}  

const Home = () => {
...
}
    \end{jscodesmall}
  \end{frame}

  \begin{frame}[fragile]
    \frametitle{Making our Button interactive}
    Let's make our button call \verb|getData| when it's pressed!
    \begin{jscodesmall}
<Button onPress={getData} />
    \end{jscodesmall}
  \end{frame}

  \section{States}

  \begin{frame}[fragile]
    \frametitle{What is a state?}
    In React/Native, we can think of a state a variable that can change over time. 
    
    We can use a state inside of components such as \verb|<View>|, which will then re-render when we change the state.

    For example, using a \verb|username| inside of a \verb|<View>|.
    \begin{jscodesmall}
<View>{username}</View>
    \end{jscodesmall}
  \end{frame}

  \begin{frame}[fragile]
    \frametitle{Making a State}
    Inside \verb|.src/pages/Home.jsx|, we'll make a state. First, import \verb|useState|. 

    \begin{jscodesmall}
import React, {useState} from 'react'
    \end{jscodesmall}
  \end{frame}
  \begin{frame}[fragile]
    \frametitle{Making a State (cont.)}
    We'll now make a state inside \verb|./src/pages/home.jsx|.
    It will go inside the \verb|Home| const, and above the \verb|return()|.
    
    \begin{jscodesmall}
const [newsItem, setNewsItem] = useState()
    \end{jscodesmall}

    Now, rename the previous \verb|newsItem| we created to \verb|newsItemExample|.

  \end{frame}

  \begin{frame}[fragile]
    \frametitle{Using the state}
    Now, let's change the \verb|getData| function to use \verb|setNewsItem()|.

    Make sure to move the function to be under the \verb|useEffect()| line of code. 
    
    \begin{jscodesmall}
const getData = async() => {
  const response = await fetch('https://newsapi.org/v2/top-headlines?country=us&apiKey=7c185ff7708e44dea7b0f22f482203d5')
  const parsedResponse = await response.json()
  const articles = parsedResponse.articles
  setNewsItem(articles)
}  
    \end{jscodesmall}
  \end{frame}


  \appendix

  \begin{frame}[standout]
    End of Lesson

    {\small Questions? Reach out at:}
    {\footnotesize karlorjalo@gmail.com}
  \end{frame}

\end{document}