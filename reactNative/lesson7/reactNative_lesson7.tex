\documentclass{beamer}
\hypersetup{pdfpagemode=FullScreen}
\usepackage{fontspec}
\usepackage{tcolorbox}
\tcbuselibrary{minted,skins}
\usepackage{graphicx}
\graphicspath{ {./images/} }

\newtcblisting{jscode}{
  listing engine=minted,
  colback=codebg,
  colframe=codebg,
  listing only,
  minted style=monokai,
  minted language=jsx,
  left=1mm,
}
\newtcblisting{jscodesmall}{
  listing engine=minted,
  colback=codebg,
  colframe=codebg,
  listing only,
  minted style=monokai,
  minted language=jsx,
  minted options={fontsize=\tiny},
  left=1mm,
}
\newtcblisting{bashcode}{
  listing engine=minted,
  colback=bashbg,
  colframe=bashbg,
  listing only,
  minted style=native,
  minted language=bash,
  minted options={linenos=true,texcl=true},
  left=1mm,
}
\definecolor{codebg}{rgb}{0.2,0.2,0.2}
\definecolor{bashbg}{rgb}{0.85,0.85,0.85}

\setsansfont{Open Sans}

\usetheme{metropolis}           % Use metropolis theme
\title{Lesson 7: Becoming a React Native native}
\subtitle{Working with Firebase}
\date{\today}
\author{Karl Steven Velasco Orjalo}
\institute{React Native Quickstart}

\begin{document}
\metroset{block=fill}
  \maketitle

  \begin{frame}{Lesson Outline}
    \setbeamertemplate{section in toc}[sections numbered]
    \tableofcontents
  \end{frame}

  \section{What is Firebase?}
  \begin{frame}[fragile]
    \frametitle{What is Firebase?}
    Firebase is a cloud services provider that Google runs. 

    The specific service we will be using in this lesson is Firestore, a no-SQL database. 
    This means that data is classified into documents and collections. 
  \end{frame}

  \section{Using Firebase}
  \begin{frame}[fragile]
    \frametitle{Taking a look at the Firestore console}
    Head to \url{console.firebase.google.com}
  \end{frame}

  \begin{frame}[fragile]
    \frametitle{Taking a look at the Firestore console (cont.)}
    Create a new project, then call it \verb|myNewProject|. 

    This name can be anything really, and it does not have to match the name of your working folder/repo.
  \end{frame}

  \begin{frame}[fragile]
    \frametitle{Taking a look at the Firestore console (cont.)}
    Now head to the button that looks like \verb|</>|, where we can `Add Firebase to your web app'. 

    Then, we can register our app as \verb|myNewProject|. Again, this doesn't matter too much. 

    Back in VSCode, in your Terminal enter the following command: 

    \begin{bashcode}
yarn add firebase
    \end{bashcode}
  \end{frame}

  \begin{frame}[fragile]
    \frametitle{Taking a look at the Firestore console (cont.)}
    Then, in the next step `Add Firebase SDK', copy the \verb|const firebaseConfig = {}| and its contents. 

    This will be used in our project. 
  \end{frame}

  \begin{frame}[fragile]
    \frametitle{Setting up Firebase in React Native}
    Back in VSCode, create a folder in the root folder named \verb|api| then create a file called \verb|firebaseConfig.js|. 

    The hierarchy should look like \verb|./api/firebaseConfig.js|.
  \end{frame}
  \begin{frame}[fragile]
    \frametitle{Setting up Firebase in React Native(cont.)}
    In \verb|./api/firebaseConfig.js|, enter the following code and replace the \verb|fireabseConfig| with what you copied previously.

    \begin{jscodesmall}
import { initializeApp } from 'firebase/app';
import {getFirestore} from "firebase/firestore";

const firebaseConfig = {
  // Place your firebaseConfig here
};

const app = initializeApp(firebaseConfig);
const db = getFirestore(app);

module.exports = db;
    \end{jscodesmall}
  \end{frame}

  \begin{frame}[fragile]
    \frametitle{Setting up a Firestore database in the Firebase console}
    In the Firebase console, find the `Firestore Database' page under the Build button on the left. 

    In the Firestore Database page, press `Create database'.
  \end{frame}
  \begin{frame}[fragile]
    \frametitle{Setting up a Firestore database in the Firebase console (cont.)}
    Next, select `Start in production mode' and choose \verb|asia-southeast1 (Singapore)| as the server location. 

    This will help Singapore users (in this case, us) access the database with lower latency.
  \end{frame}
  \begin{frame}[fragile]
    \frametitle{Setting up a Firestore database in the Firebase console (cont.)}
    Press `Start Collection' and call it testCollection. 
    Then, give the first document an ID of `testDocument'.

    Make two fields of type \verb|string|, named \verb|title| and \verb|body|. 
    Inside the values, give set the \verb|title| to \verb|testTitle| and \verb|body| to \verb|testBody|. 
  \end{frame}

  \begin{frame}[fragile]
    \frametitle{Getting our document into myNewProject}
    We'll now work with Firestore in our Account page. Import the \verb|databaseConfig| into \verb|./src/pages/Account.jsx|.

    \begin{jscodesmall}
import { initializeApp } from 'firebase/app';
import { getFirestore } from "firebase/firestore";

const firebaseConfig = {
  // Place your firebaseConfig here
};

const app = initializeApp(firebaseConfig);
const db = getFirestore(app);

module.exports = db;
    \end{jscodesmall}
  \end{frame}

  \begin{frame}[fragile]
    \frametitle{Getting our document into myNewProject (cont.)}
    Inside \verb|./src/pages/Account.jsx|, we now need to import a couple things.
    Try typing it out, not copying. 

    \begin{jscodesmall}
import React, {useEffect, useState} from 'react'
import db from '../../api/firebaseConfig.js'
import {collection, onSnapshot} from "firebase/firestore"
import {View, FlatList, Text} from 'react-native'
    \end{jscodesmall}
  \end{frame}

  \begin{frame}[fragile]
    \frametitle{Getting our document into myNewProject (cont.)}
    Now, let's change the displayed contents of the \verb|<View />| into a \verb|<FlatList />|. 

    \begin{jscodesmall}
return (
  <View>
    <FlatList
      data={data}
      renderItem={({ item }) => <Text>{item.title}</Text>}
    />
  </View>
)
    \end{jscodesmall}
  \end{frame}

  \begin{frame}[fragile]
    \frametitle{Getting our document into myNewProject (cont.)}
    Then, create a \verb|data| state. 

    \begin{jscodesmall}
const [data, setData] = useState()
    \end{jscodesmall}
  \end{frame}
  
  \begin{frame}[fragile]
    \frametitle{Getting our document into myNewProject (cont.)}
    To pre-load the data and receive realtime updates, we will be using the \verb|useEffect| hook and Firestore's \verb|onSnapshot()|. 

    \begin{jscodesmall}
useEffect(() => {
  const getDocuments = () => {

    const dbRef = collection(db, 'testCollection')
  
    onSnapshot(dbRef, (docsSnap) => {
      var docsArr = []

      docsSnap.forEach(doc => {
        docsArr.push(doc.data())
        setData(docsArr)
      })
    })
  }

  return getDocuments()
},[])
    \end{jscodesmall}
  \end{frame}

  


  \appendix

  \begin{frame}[standout]
    End of Lesson

    {\small Questions? Reach out at:}
    {\footnotesize karlorjalo@gmail.com}
  \end{frame}

\end{document}