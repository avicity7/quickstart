\documentclass{beamer}
\hypersetup{pdfpagemode=FullScreen}
\usepackage{fontspec}
\usepackage{tcolorbox}
\tcbuselibrary{minted,skins}
\usepackage{graphicx}
\graphicspath{ {./images/} }

\newtcblisting{jscode}{
  listing engine=minted,
  colback=codebg,
  colframe=codebg,
  listing only,
  minted style=monokai,
  minted language=jsx,
  left=1mm,
}
\newtcblisting{jscodesmall}{
  listing engine=minted,
  colback=codebg,
  colframe=codebg,
  listing only,
  minted style=monokai,
  minted language=jsx,
  minted options={fontsize=\tiny},
  left=1mm,
}
\newtcblisting{bashcode}{
  listing engine=minted,
  colback=bashbg,
  colframe=bashbg,
  listing only,
  minted style=native,
  minted language=bash,
  minted options={linenos=true,texcl=true},
  left=1mm,
}
\definecolor{codebg}{rgb}{0.2,0.2,0.2}
\definecolor{bashbg}{rgb}{0.85,0.85,0.85}

\setsansfont{Open Sans}

\usetheme{metropolis}           % Use metropolis theme
\title{Lesson 3: Exploring React Native}
\subtitle{Styling and Positioning}
\date{\today}
\author{Karl Steven Velasco Orjalo}
\institute{React Native Quickstart}

\begin{document}
\metroset{block=fill}
  \maketitle

  \begin{frame}{Lesson Outline}
    \setbeamertemplate{section in toc}[sections numbered]
    \tableofcontents
  \end{frame}

  \section{Customizing our Bottom Bar}
  \begin{frame}[fragile]
    \frametitle{Adding Icons to the Bottom Bar}
    Our Bottom Bar looks plain with triangles, let's add icons!

    For all the slides regarding the Bottom Bar icons, we'll be working in \verb|./App.js|.
  \end{frame}
  \begin{frame}[fragile]
    \frametitle{Adding Icons to the Bottom Bar (cont.)}
    First, let's add an external library giving us icons. 

    Add \verb|MaterialCommunityIcons| just below our other external dependencies: 

    \begin{jscode}
import { createBottomTabNavigator } from "@react-navigation/bottom-tabs";
import { NavigationContainer, StackActions } from "@react-navigation/native";
import { SafeAreaProvider } from "react-native-safe-area-context";
import MaterialCommunityIcons from "@expo/vector-icons/MaterialCommunityIcons";
    \end{jscode}
  \end{frame}
  \begin{frame}[fragile]
    \frametitle{Adding Icons to the Bottom Bar (cont.)}
    Next, let's give icons to our Bottom Bar.

    Make changes to <TabNavigator> as below:

    \begin{jscodesmall}
  <Tab.Navigator
    screenOptions={({ route }) => ({
      tabBarIcon: ({ color, size }) => {
        let iconName;
        if (route.name === "Home") {
          iconName = "home";
        } else if (route.name === "Account") {
          iconName = "account";
        }

        return (
          <MaterialCommunityIcons
            name={iconName}
            size={30}
            color={color}
            style={{ height: 30 }}
          />                              
        );
      }
    })}
  >
    \end{jscodesmall}
  \end{frame}
  \begin{frame}[fragile]
    \frametitle{Choosing different icons}
    For the library \verb|MaterialCommunityIcons|, the icon names are taken from \url{https://pictogrammers.com/library/mdi/}.

    \vspace{0.5cm}
    Experiment with different icons!
  \end{frame}

  \section{Styling a Page}
  \begin{frame}[fragile]
    \frametitle{Centering our Page Text}

    As we can see on both the Home and Account page, the Text is at an awkward position at the top left. 

    Under \verb|./src/styles/|, create a file called \verb|global.js|.
  \end{frame}
  \begin{frame}[fragile]
    \frametitle{Centering our Page Text (cont.)}

    In \verb|./src/styles/global.js|, add the following code: 

    \begin{jscode}
import { StyleSheet, Dimensions } from "react-native";
const globalStyles = StyleSheet.create({
})
export default globalStyles
    \end{jscode}
  \end{frame}
  \begin{frame}[fragile]
    \frametitle{Centering our Page Text (cont.)}

    We will now define \verb|globalStyles.container| as below:

    \begin{jscodesmall}
const globalStyles = StyleSheet.create({
  container: {
    flex: 1,
    backgroundColor: "white",
    padding: 16,
    alignItems: "center"
  }
})
    \end{jscodesmall}
  \end{frame}
  \begin{frame}[fragile]
    \frametitle{Centering our Page Text (cont.)}

    Now, we will use the style we defined in \verb|./src/pages/home.jsx|.

    Import \verb|globalStyles| just below our dependency imports like below: 

    \begin{jscodesmall}
import React from 'react'
import {View, Text} from 'react-native'

import globalStyles from '../styles/globalStyles'
    \end{jscodesmall}
  \end{frame}
  \begin{frame}[fragile]
    \frametitle{Centering our Page Text (cont.)}

    Then, edit \verb|<View>| to take in the style:

    \begin{jscodesmall}
<View style={globalStyles.container}>
  <Text>This is the Home page.</Text>
</View>
    \end{jscodesmall}
  \end{frame}

  \begin{frame}[fragile]
    \frametitle{Verifying our change}

    Now, run the app on Expo Go. 
    You should see the difference between the \verb|Home| page and \verb|Account| page. 

    If you haven't noticed yet, styling in React Native maps its methods to CSS styling methods!
  \end{frame}

  \begin{frame}[fragile]
    \frametitle{Changing the appearance of Text}

    Try making a new style definition under \verb|globalStyles| in \verb|./src/styles/global.js| for the Text in the \verb|Home| page!
  \end{frame}

  \appendix

  \begin{frame}[standout]
    End of Lesson

    {\small Questions? Reach out at:}
    {\footnotesize karlorjalo@gmail.com}
  \end{frame}

\end{document}